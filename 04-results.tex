\chapter{Results and the Future}
\label{cha:results}
In this chapter, the results are analysed with regards to performance and correctness. Furthermore the limitations of this solution are discussed as well as possible improvements for the future.

\section{Assessing Correctness}

\section{Testu01 Compatibility}

\section{Limitations and Future Work}
\subsection{Generator setup time}
Currently, the biggest limitation of this solution is the fact that a test generator is initialised every time a test is run. Since the tests run in separate processes, a new generator is set up for each one.
For trivial generators, this cost might be insignificant, but for generators that are simulations of hardware, it can overshadow the performance gains of parallelisation. The following equations can help illustrate this point:

Linear test suite completion time: N * AverageTestTime + GenSetupTime
Parallel test suite completion time: N * (AverageTestTime + GenSetupTime) / NumbOfCores
Solving for GenSetupTime:

If this equation holds, then the performance of the parallel suite will be at worst the same as the serial one.
The number of cores of modern desktops and laptops ranges between two and eight. Using these numbers, estimates for the maximum efficient parallel GenSetupTime can be acquired. Since systems with more cores than eight were used in the actual measurements, estimations are included for those too.
Number of cores
2 4 6 8
smallCrush
Crush
BigCrush

\subsection{Lack of flexibility}

\subsection{Windows support}
The original test suites could be compiled for Windows. The Testu01 homepage provides precompiled Windows binaries which can be used to run the tests.

However, despite this project being theoretically still portable and working on Windows, it has not been tested for compatibility. This is due to lack of time and lack of knowledge and experience with working on Cygwin on Windows.

\subsection{File Generators}
Currently the three Crush suites allow for use of files as random number generators. This functionality has not been tested at all and was not taken under consideration when building the parallel Testu01. As a result, the best approximation that can be given for using the parallel suites with a file generator is simply undefined behaviour and should not be used at all.

However, given the nature of files, it would quite probably be possible to extend the parallelism for these generators as well without major restructuring of the code base.
